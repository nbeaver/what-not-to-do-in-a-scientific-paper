\documentclass[12pt,letterpaper]{article}
\usepackage[utf8]{inputenc}
\usepackage{amsmath}
\usepackage{amsfonts}
\usepackage{amssymb}
\usepackage[pdftex]{hyperref}
\author{Nathaniel Beaver}
\title{What I won't do in a scientific paper.}
\begin{document}

\maketitle

\begin{itemize}
\item I will not write anything like ``It is easy to see that...'' If it is easy to see, then it is unnecessary to say so. If it is not easy to see, I have needlessly irritated anyone reading the paper.
\item I will not make a log-log plot of my data and then \href{http://vserver1.cscs.lsa.umich.edu/~crshalizi/notebooks/power-laws.html}{conclude I have a power law} based on a least-squares linear fit. 
\item I will not cite a figure in another paper without giving the page number and figure number, unless the result is in the abstract of the other paper.
\item I will not cite a result in another paper without quoting the relevant sentence or paragraph. I will do this to avoid confusion and provide context even when it may weaken my thesis.
\item When I cite a prior result, I will cite the first paper or papers to report a result, not a more recent paper I happened to read which cites the original papers.
\item If there is disagreement in the literature about a result, I will cite papers representative of literature, not only the one which supports my thesis.
\item I will not use \href{http://matt.might.net/articles/shell-scripts-for-passive-voice-weasel-words-duplicates/}{weasel} \href{http://en.wikipedia.org/wiki/Weasel_words}{words} like `most', `fairly', or `very'. They are warning signs of a lack of diligence and clarity.
\item I will not use an acronym without defining it and displaying the definition prominantly in a list of acronyms. This will save the people reading my article time and frustration hunting for definitions of acronyms buried in the text.
\item When I cite calculations or numeric results from another paper, I will either cite them verbatim or show my calculations on the verbatim results. I will not perform calculations, show the result, and then cite the other paper. This will prevent anyone reading my paper from having to guess how I calculated my result and will make it apparent whether or not I made a mistake.
\item When writing and revising papers, I will only discuss figures which are actually shown in my paper. If figures or tables are pushed into ``supporting material'', I will not refer to them in the published paper, but instead discuss their implications in the supporting material.
\end{itemize}
\end{document}
