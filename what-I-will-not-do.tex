\documentclass[12pt,letterpaper]{article}
\usepackage[utf8]{inputenc}
\usepackage{amsmath}
\usepackage{amsfonts}
\usepackage{amssymb}
\usepackage[pdftex]{hyperref}
\author{Nathaniel Beaver}
\title{What I won't do in a scientific paper.}
\begin{document}

\maketitle

\begin{itemize}
\item I will not write anything like ``It is easy to see that...'' If it is easy to see, then it is unnecessary to say so. If it is not easy to see, I have needlessly irritated anyone reading the paper.
\item I will not make a log-log plot of my data and then \href{http://vserver1.cscs.lsa.umich.edu/~crshalizi/notebooks/power-laws.html}{conclude I have a power law} based on a least-squares linear fit. 
\item I will not cite a result in another paper without giving the page number or figure number, unless the result is in the abstract of the other paper. If there is any possibility of confusion or misinterpretation of the other paper, I will quote the relevant sentence or paragraph of the other paper.
\item I will cite the first paper or papers to report a result, not a more recent paper I happened to read which cites the original papers.
\item If there is disagreement in the literature about a result, I will cite papers representative of literature, not only the one which supports my thesis.
\item I will not use \href{http://matt.might.net/articles/shell-scripts-for-passive-voice-weasel-words-duplicates/}{weasel} \href{http://en.wikipedia.org/wiki/Weasel_words}{words} like `most', `fairly', or `very'.
\item I will not use an acronym without defining it and displaying the definition prominantly in a list of acronyms. This will save the people reading my article much time and frustration hunting for definitions of acronyms buried somewhere in the paper.
\end{itemize}
\end{document}
